\documentclass{article}
\usepackage[brazil]{babel}
\usepackage[utf8]{inputenc}
\usepackage{url}
\usepackage{multicol}
\usepackage[bottom=2.5cm,top=2.5cm,left=2.5cm,right=2.5cm]{geometry}
\usepackage{graphicx}
\usepackage{float}
\usepackage{caption}
\usepackage[dvipsnames]{xcolor}
%\usepackage{subcaption}

\usepackage{setspace}
\usepackage{indentfirst}
\usepackage{mathtools}
\usepackage{amsmath}
\usepackage{subfigure}

\title{Exercício Programa (EP) 2\\
    \large \texttt{[MAC0460]}}

\author{Julia Leite\\
    \large \texttt{NUSP: 11221797}}

\begin{document}
    
\maketitle

\tableofcontents

\section{Introdução}

Este projeto tem como objetivo construir uma ferramenta que classifique \textit{pokemons} em \textit{water} ou \textit{normal} para auxiliar em uma expedição do prof. Carvalho. Como apenas o Pikachu acompanha a jornada e não pode sofrer danos, o conjunto de informações disponíveis para análise é limitado.

Para encontrar o melhor classificador, comparamos o desempenho de diferentes modelos de aprendizado de máquina: Regressão Logística, Support Vector Machines (SVM), Árvores de Decisão e Random Forest.

\section{Metodologia}

\subsection{Dados}

O \textit{dataset} utilizado nesse projeto foi retirado de \url{https://www.kaggle.com/datasets/rounakbanik/pokemon} e contém diversas informaçoões sobre os \textit{pokemon} 

\section{Resultados}

\section{Discussão}

\section{Conclusão}

\end{document}